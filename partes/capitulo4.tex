\chapter{Marco Metodológico}
\label{capitulo4}
\lhead{Capítulo 4. \emph{Marco Metodológico}}

A continuación se procederá a describir el proceso de gestión
del proyecto de pasantía.

El equipo de Everis encargado de mantener las arquitecturas tecnológicas
de BBVA trabaja bajo la metodología de desarrollo ágil SCRUM para el
desarrollo de sus tareas. El estudiante desarrolló el proyecto bajo
esta metodología que define el desarrollo de software como un proceso
incremental e iterativo, con entregables parciales y frecuentes al
cliente (BBVA). La definición de iteraciones, llamadas Sprints, de baja duración
le brindan a esta metodología la ventaja de ser un proceso bastante flexible y
adaptativo a los posibles cambios de requerimientos que puedan ocurrir en el proyecto.

SCRUM se define mediante los elementos descritos en este capítulo.

\section{Sprint}
En SCRUM las iteraciones en las cuales se lleva acabo el desarrollo reciben el nombre
de \emph{Sprint}. Para que la metodología sea eficaz se deben definir los sprints con
una duración relativamente corta, entre 2 o 3 semanas. Ésta es la base sobre la cual
se obtienen las ventajas de flexibilidad y adaptabilidad de la metodología.

En el proyecto de pasantía los Sprint tuvieron una duración de 2 semanas.

\section{Roles}

\subsection{Product Owner}
Es el encargado de representar los intereses del cliente y es el responsable de
que el equipo desarrolle artefactos con valor para el proyecto. Tiene la responsabilidad
de crear y priorizar las tareas a realizar por el equipo en cada iteración.

En estas pasantías el rol de \emph{Product Owner} lo llevó a cabo el Ing. Rubén
Maldonado.

\subsection{Scrum Master}
El Scrum Master o facilitador es el encargado de velar por que el equipo de
desarrollo no tenga ningún tipo de impedimento a la hora de realizar sus tareas.
Además es el encargado de que se sigan las pautas establecidas por la metodología
de desarrollo.

Para el proyecto el Ing. Felipe Díaz asumió el papel de \emph{Scrum Master}.

\subsection{Equipo}

El equipo de desarrollo es el conjunto de personas encargadas de realizar las tareas
especificadas en cada sprint. La metodología SCRUM no define restricciones sobre como
debe organizarse internamente el equipo, o en otras palabras, se le da la libertad al
equipo de decidir como debe organizarse para cada sprint.

El proyecto de pasantías fue realizado exclusivamente por el estudiante, por lo que
en este caso sería el único integrante del equipo.


\section{Actividades}

Para garantizar el correcto funcionamiento de la metodología se realizan frecuentemente
las siguientes actividades:

\subsection{Scrum diario}
Es una reunión corta, con duración máxima de 15 minutos, que se realiza todos los días
del sprint. El objetivo de esta reunión es que cada miembro del equipo exponga cual
es el estado de sus tareas del sprint.

\subsection{Planificación de Sprint}
Al inicio de cada Sprint se deben reunir el Product Owner, el Scrum Master y el equipo
para tomar las decisiones pertinentes sobre el sprint que va a comenzar. En esta actividad
se define:

\begin{itemize}
  \item Cuales son las tareas a realizar.
  \item Definir cuanto tiempo tomará cada tarea.
\end{itemize}

\subsection{Cierre de Sprint}
Una vez finalizado el sprint se revisa cuales de las tareas asignadas en la planificación
fueron terminadas y se presentan los resultados de las tareas terminadas a los posibles
interesados.

\subsection{Retrospectiva de Sprint}

Esta actividad se lleva acabo al finalizar el sprint. Tiene como finalidad revisar como
se desarrolló la metodología en el sprint y se busca mejorar y afinar el proceso.

\section{Artefactos}

Para la gestión del proyecto se utilizan ciertos documentos que permiten visualizar
claramente la planificación y el estado actual de las tareas dentro de la metodología.
A continuación se explican los artefactos utilizados.

\subsection{Historia de usuario}

Las historias de usuario representan los requisitos a cumplir por el proyecto. Dentro
de la metodología SCRUM deben ser escritas en una o dos frases sencillas y utilizando
el lenguaje común del cliente. Son escritas y manejadas por el Product Owner. Cada
historia de usuario es luego subdividida en tareas, las cuales deben tener una duración
corta que permita su realización dentro de un sprint.

\subsection{Product Backlog}

Es un documento de alto nivel al cual tienen acceso todos los integrantes del proyecto,
pero que sólo puede ser modificado por el Product Owner. En el se muestran todas las
historias de usuario priorizadas según su valor para el negocio.

\subsection{Sprint Backlog}
Es el artefacto utilizado para recolectar el conjunto de tareas a realizarse durante
el sprint en curso. Durante la planificación del sprint se seleccionan las tareas a
realizar y éstas son agregadas al Sprint Backlog para que luego sea gestionado por el
equipo, marcando cuales tareas están en progreso, cuales están listas y cuales por hacer.
