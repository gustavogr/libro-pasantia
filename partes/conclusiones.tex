\chapter{Conclusiones y Recomendaciones}
\label{conclusiones}
\lhead{\emph{Conclusiones y Recomendaciones}}

El proyecto de pasantías fué culminado exitosamente y dentro del tiempo
estimado, produciendo sendas mejoras para las arquitecturas tecnológicas
Thin2 y ePhoenix usadas en el banco español BBVA. En el caso de ePhoenix
se desarrolló un \emph{bundle} que permitiera a los equipos de desarrollo
realizar conexiones a bases de datos manejadas con PostgreSQL y Mirosoft SQL
Server. Para la arquitectura de aplicaciones web Thin2 se desarrolló un
componente de Angular que le permite a los desarrolladores construir tablas
de datos de una manera sencilla y clara.

La pasantía fue gestionada con el método de desarrollo ágil SCRUM, el
cual le brinda al proyecto una estructura clara y organizada de como realizar
el trabajo, permitiendo a su vez que esta planificación sea flexible y fácilmente
adatable a los cambios de requerimientos que se presentaron durante el proyecto.
El trabajo fue dividido en 10 sprints con una duración de dos semanas cada uno.
Cabe destacar a pesar de que la metodología está diseñada para ser llevada acabo
por equipos de desarrollo integrado por varias personas, en este caso el equipo
solo estuvo integrado por el estudiante debido a que la pasantía fue realizada
de manera individual.

Tanto en ePhoenix como en Thin2 se trabaja bajo la filosofía de desarrollo orientado
a componentes. Al seguir los lineamientos de esta estrategia los componentes desarrollados
presentan una alta cohesión por mantener toda la funcionalidad
relacionada en un sólo componente, y también presentan un bajo acoplamiento debido
a que las piezas de software desarrolladas no dependen del funcionamiento interno de otros
componentes. Éstas son características altamente deseadas en el desarrollo de software, ya que
contribuyen con la mantenibilidad y el fácil entendimiento del código.

Para futuros trabajos se recomienda mantener siempre la documentación de las soluciones
desarrolladas actualizadas. De esta manera las personas que vayan a utilizar o modificar
el software pueden entrar en contexto rápidamente, facilitando la labor y la productividad
de los proyectos. Además se recomienda la utilización de metodologías de desarrollo ágil,
como SCRUM, ya que plantean una manera clara de organizar el trabajo sin sacrificar
la flexibilidad necesaria de un equipo de desarrollo para responder ante posibles cambios de
requerimientos.
