\chapter{Marco Teórico}
\label{capitulo2}
\lhead{Capítulo 2. \emph{Marco Teórico}}

A continuación se detallan los aspectos teóricos sobre los cuales se basan
tanto las herramientas utilizadas en la pasantía como la solución propuesta
en este proyecto:

\section{API}
Un API (\emph{Application Programming Interface}) es un conjunto de comandos,
funciones, protocolos o métodos que le permiten a un programa comunicarse
con un sistema externo o librería. Las APIs permiten que varios sistemas
informáticos (Sistemas Operativos, aplicaciones, etc.) se comuniquen sin la
necesidad de exponer la funcionalidad interna de cada sistema. A su vez esto
le permite a los desarrolladores utilizar funcionalidades de otras aplicaciones
sin la necesidad de reescribir código (\cite{API}).

\section{Bases de datos relacional}

Una base de datos es un medio para almacenar y recuperar información. En términos simples
una base de datos relacional es aquella en la que la información es almacenada en tablas con
filas y columnas. Se le hace referencia a las tablas como relaciones ya que representa una
coleccion de objetos del mismo tipo. La habilidad de recuperar información relacionada de una
tabla es la base para el término Base de datos relacional (\cite{RELACIONAL}).

\section{Desarrollo Orientado a Componentes}

Este método de diseño de software establece que a la hora de construir un sistema
informático, las funcionalidades de éste sean separadas en unidades más pequeñas
llamadas componentes. Un componente es una pieza de software que ofrece a través
de una interfaz un servicio predefinido y que es capaz de comunicarse con otros
componentes (\cite{COMPONENT}). Esta estrategia de desarrollo trae como resultado
software que puede ser reutilizado con mayor facilidad y que es fácilmente escalable
agregando futuros componentes que agreguen nuevas funcionalidades. Además, facilita la
mantenibilidad del código, debido a que la encapsulación de las funcionalidades permite
detectar con mayor facilidad de donde proviene una falla.

\section{Modelo Vista Controlador}
Modelo-Vista-Controlador (MVC) es un patrón de diseño de software, especialmente útil
para diseñar interfaces de usuario. Se basa en la separación de los datos y lógica de
negocio, la interfaz gráfica a mostrar y el módulo encargado de atender y responder a
los eventos generados por el usuario en tres componentes distintos (\cite{MVC}). Estos
componentes quedan definidos de la siguiente manera:

\begin{itemize}
  \item \textbf{Modelo}: se encarga de representar los datos que maneja el sistema y
  la funcionalidad asociada que tengan estos modelos.
  \item \textbf{Vista}: está a cargo de mostrar la representación visual del modelo
  para que el usuario interactúe fácilmente con el sistema.
  \item \textbf{Controlador}: es el módulo encargado de coordinar las vistas con el modelo.
  Para ello atiende los eventos que pueda generar el usuario para luego actualizar
  adecuadamente la vista y el modelo.
\end{itemize}

Este patrón de arquitectura de software se ha vuelto bastante común en el desarrollo
de interfaces de usuario por su ventajas en cuanto a reutilización y mantenibilidad
del código.
