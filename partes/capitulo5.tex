\chapter{Desarrollo del Proyecto}
\label{capitulo4}
\lhead{Capítulo 4. \emph{Desarrollo del Proyecto}}

En este capítulo se detallan las actividades realizadas para el
desarrollo del proyecto de pasantías. El proyecto fue realizado
mediante la metodología de desarrollo SCRUM descrita en el capítulo
anterior, con \emph{Sprints} de dos semanas de duración, por lo
que la pasantía quedó dividida en 10 \emph{sprints}. El proyecto
se divide en dos grandes fases: el desarrollo del evolutivo para ePhoenix
y el desarrollo del evolutivo para Thin2. A su vez, cada etapa contempla
las fases de levantamiento de información, análisis, diseño, implementación,
pruebas y despliegue de los artefactos desarrollados. A continuación se
expone por \emph{Sprint} los objetivos y las actividades realizadas.

\section{Primer Sprint}

\subsection{Objetivos}
\begin{itemize}
\item Estudio de las herramientas utilizadas en el banco.
\end{itemize}
\subsection{Actividades}
\subsubsection{Leer documentación de las arquitecturas}
Para ambas arquitecturas (ePhoenix y Thin2) existe documentación interna del BBVA
para su uso, la cual fué leída por el estudiante.
\subsubsection{Asistir a formaciones sobre el uso de las arquitecturas}
Dentro del banco se ofrecen formaciones a los nuevos equipos de desarrollo
sobre como utilizar las arquitecturas del banco para desarrollar aplicaciones con
ellas. El pasante asisitió a una formación para cada arquitectura para tener
una perspectiva más práctica sobre el uso de las plataformas.
\subsubsection{Realizar tutoriales}
Para asentar los conocimientos adquiridos en las actividades previas se propuso
realizar tutoriales sobre las herramientas a utilizar. Para Thin2 se realizó un
tutorial de Angular (el \emph{framework} sobre el cual se basa Thin2) y para ePhoenix
se siguieron las guías mencionadas en la documentación para realizar un bundle sencillo
de procesamiento por lotes.

\section{Segundo Sprint}

\subsection{Objetivos}
\begin{itemize}
  \item Levantar información sobre necesidades en ePhoenix.
  \item Detectar una mejora o carencia en la arquitectura.
  \item Diseñar una solución para dicha carencia.
\end{itemize}
\subsection{Actividades}
\subsubsection{Reunión con distintos equipos de desarrollo}
Se concretaron reuniones breves con los distintos equipos de desarrollo del banco
para que pudieran plantear que características piensan que se podrían mejorar o agregar a la
arquitectura ePhoenix.
\subsubsection{Análisis de la información recolectada}
Tomando en cuenta los comentarios recibidos en las reuniones informativas se llegó
a la conclusión que existía la necesidad de poderse conectar a bases de datos de otros
manejadores distintos a Oracle.
\subsubsection{Diseño de solución para conectar con otros manejadores de Base de datos}
Una vez analizados los \emph{bundles} que toman parte en la realización de conexiones a bases
de datos se determinó que se necesita crear un bundle para cada driver de conexión a base de
datos. Después de creados estos componentes, se debe implementar un \emph{bundle} que permita
la inclusión de estos drivers en el componente de conexión a base de datos. Finalmente se deberá
modificar la interfaz de administración de ePhoenix para permitirle a los usuarios
crear conexiones de distintos proveedores de base de datos.

\section{Tercer Sprint}

\subsection{Objetivos}
\begin{itemize}
  \item Implementar los \emph{bundles} necesarios para realizar conexiones.
\end{itemize}
\subsection{Actividades}
\subsubsection{Convertir los drivers de PostgreSQL y Microsoft SQL en \emph{bundles}}
Para esto se descargaron los drivers de PostgreSQL y Microsoft SQL disponibles en
internet y se realizaron las modificaciones necesarias para que sea reconocido por
OSGI como un bundle.
\subsubsection{Implementar los \emph{bundles} de tipo fragmento}
Luego, el pasante pasó a crear los \emph{bundles} responsables de comunicar los
drivers de los manejadores de base de datos con el componente que crea las conexiones
a base de datos.
\subsubsection{Probar las conexiones creadas}
Una vez creados los artefactos mencionados en las actividades pasadas se procedió
a crear un \emph{bundle} de prueba que realizara una conexión con una base de datos
PostgreSQL y una manejada con Microsoft SQL Server y reportara éxito en caso de
realizar la conexión satisfactoriamente.

\section{Cuarto Sprint}

\subsection{Objetivos}
\begin{itemize}
  \item Modificar los bundles de la interfaz de administración para crear nuevas
  conexiones.
\end{itemize}
\subsection{Actividades}
\subsubsection{Agregar un campo desplegable que permita elegir el tipo de base de datos}
Los equipos de desarrollo cuentan con una interfaz administrativa para la creación y configuración
de conexiones para sus aplicaciones. En las figuras del apéndice \ref{apendiceA} se pueden ver las
modificaciones realizadas a la interfaz.
\subsubsection{Modificar el bundle que crea la configuracion para la conexión}
Adicionalmente existe otro \emph{bundle} que es el encargado de una vez configurada la conexión
en el tablero administrativo, cree los archivos de configuración necesarios para establecer
la conexión a la base de datos.

\section{Quinto Sprint}

\subsection{Objetivos}
\begin{itemize}
  \item Realizar pruebas sobre los bundles
  \item Desplegar los bundles implementados para su utilización.
\end{itemize}
\subsection{Actividades}
\subsubsection{Realización de pruebas}
Para garantizar el correcto funcionamiento de los bundles desarrollados
se probó que se realizaran correctamente las conexiones a base de datos en distintos
entornos y con distintas bases de datos.
\subsubsection{Desplegar con Dimensions los bundles}
Una vez implementados y probados los componentes necesarios, se procedió a publicar
el desarrollo a través de la herramienta \emph{Dimensions} para que esté disponible
en los servidores de producción para su uso por los equipos de desarrollo.
\subsubsection{Documentar la solución planteada}
Se actualizó la documentación interna de la arquitectura ePhoenix para que reflejara
la posibilidad de crear conexiones de bases de datos de otros proveedores con pasos
detallados de como realizar estas conexiones.

\section{Sexto Sprint}

\subsection{Objetivos}
\begin{itemize}
  \item Levantar información sobre necesidades en Thin2.
  \item Detectar una mejora o carencia en la arquitectura.
  \item Diseñar una solución para dicha carencia.
\end{itemize}
\subsection{Actividades}
\subsubsection{Reunión con distintos equipos de desarrollo}
Al igual que con ePhoenix se llevaron acabo reuniones con los desarrolladores
de aplicaciones web para que pudieran dar recomendaciones sobre que funcionalidad
mejoraría su experiencia a la hora de crear aplicaciones.
\subsubsection{Análisis de la información recolectada}
Después de recolectada y analizada la información se hizo notar que una operación
bastante común a la hora de realizar aplicaciones web para el banco era mostrar
información en tablas. Este proceso a su vez suele ser tedioso y con resultados
bastante heterogéneos. Es por esto que se decidió crear un componente para Thin2
que facilitara la creación de tablas de datos.
\subsubsection{Diseño de solución para facilitar la creación de tablas de datos}
Para facilitar la creación de tablas se decidió crear un componente reutilizable
de Angular. Este componente, llamado \emph{th2-grid-component}, recibiría como
parámetros de entrada un objeto JSON que describe las columnas a crear, un objeto
JSON con la información a mostrar en la tabla, y una función opcional que determina
el comportamiento a realizar si se hace click en una celda de la tabla.

\section{Séptimo Sprint}

\subsection{Objetivos}
\begin{itemize}
  \item Implementar versión básica del componente de tablas.
\end{itemize}
\subsection{Actividades}
\subsubsection{Implementar primera versión del componente}
En esta primera iteración se le dió mayor prioridad a desarrollar las características
básicas del componente, que son poder crear tablas con un formato sencillo y que muestren
la información deseada.


\section{Octavo Sprint}

\subsection{Objetivos}
\begin{itemize}
  \item Implementar segunda fase del componente.
\end{itemize}
\subsection{Actividades}
\subsubsection{Implementar mejora que permita darle formato a las tablas.}
Se mejoró el componente para que pudiera recibir clases HTML para cada columna
para que el desarrollador pueda darle formatos específicos a las columnas y a las celdas
de la tabla.
\subsubsection{Permitir que el componente reciba por parámetro la función a ejecutar.}
Luego se pasó a implementar la útlima funcionalidad deseada, que permite que el
componente reciba la función a ejecutar si se hace click sobre una celda de la tabla.

\section{Noveno Sprint}

\subsection{Objetivos}
\begin{itemize}
  \item Desplegar el componente en el repositorio de artefactos del banco.
  \item Documentar y divulgar la creación del nuevo componente para Thin2.
\end{itemize}
\subsection{Actividades}
\subsubsection{Desplegar el componente desarrollado a través de Dimensions}
El componente implementado fue desplegado a través de Dimensions para que esté
disponible en el repositorio de artefactos interno del banco.
\subsubsection{Actualizar documentación sobre la arquitectura}
Se llevó acabo la actualización de la documentación interna del banco sobre
Thin2 para agregar un inciso sobre el nuevo componente implementado con
una breves instrucciones sobre como utilizar dicha funcionalidad.

\section{Décimo Sprint}

\subsection{Objetivos}
\begin{itemize}
  \item Crear una aplicación de ejemplo que muestre el uso del componente de tablas.
\end{itemize}
\subsection{Actividades}
\subsubsection{Desarrollar una pequeña aplicación web de ejemplo}
Para facilitar el uso del componente de tablas por parte de los equipos de
desarrollo el pasante implementó una pequeña aplicación web que ejemplificara
como utilizar el componente desarrollado. En el apéndice \ref{apendiceB} se muestra parte
del código utilizado para crear una tabla y el resultado correspondiente en el navegador.
