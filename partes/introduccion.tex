\chapter*{Introducción}
\label{intro}
\lhead{\emph{Introducción}}
\addcontentsline{toc}{chapter}{Introducción}
% Organización del trabajo
% Se describe brevemente qué se hace en cada capítulo
El presente informe detalla el proceso de desarrollo del software
necesario para permitir la conexión a bases de datos manejadas con
PostgreSQL y MicrosoftSQL Server en la arquitectura ePhoenix
y la implementación de un componente que permita crear fácilmente
tablas en una interfaz web para la arquitectura Thin2.

El documento se organiza en capítulos de la siguiente forma: la
Introducción al proyecto de Pasantías; en el Capítulo 1 se menciona
el contexto empresarial y el cargo ocupado por el pasante en la
consultora Everis durante la pasantía; en el Capítulo 2 se explican
los fundamentos teóricos necesarios para la comprensión del problema y
la solución planteada; en el Capítulo 3 se detallan las tecnologías
implicadas en el proceso de desarrollo del proyecto; en el Capítulo 4 se
explica el funcionamiento de la metodología de gestión de proyectos utilizada
(SCRUM); en el Capítulo 5 se describen las actividades realizadas por el
estudiante; finalmente se le dan al lector las Conclusiones y recomendaciones.

% Descripción del problema, de lo general hacia lo específico
\section*{Antecedentes}
El banco BBVA es una de las entidades líderes de España, con más de 10
millones de clientes y cerca de 30.000 empleados, prestando servicios
financieros a través de su red de 3200 oficinas (\cite{BBVA}). Dentro de
su área de tecnología cuenta con diferentes plataformas y arquitecturas
utilizadas con el objetivo de reducir el tiempo y homogeneizar el desarrollo.

\section*{Justificación e importancia}
Una de las constantes a la hora de desarrollar software es que los
requerimientos pueden cambiar a lo largo del tiempo. Ésto también
es cierto para los empleados encargados de desarrollar aplicaciones
dentro del banco BBVA. Es por esto que es importante que las arquitecturas
que se utilicen para desarrollar soluciones tecnológicas, en específico
las arquitecturas ePhoenix y Thin2, se mantengan actualizadas y se mejoren
para poder satisfacer todas las necesidades de los desarrolladores.

\section*{Planteamiento del problema}

Con la finalidad de mantener las arquitecturas utilizadas en el banco lo
más actualizadas posibles es necesario un proceso constante de mejoras de
las mismas. Este proceso incluye una fase de levantamiento de requerimientos
para identificar fallas o posibles mejoras de las plataformas utilizadas. Luego,
una etapa de diseño de solución que solvente la deuda técnica detectada, para
finalmente pasar al desarrollo, prueba y despliegue de dicha solución.

Para contribuir con dicha actualización de ambas arquitecturas el pasante realizará
una iteración de las etapas mencionadas sobre cada una de las plataformas para
así producir un evolutivo para ePhoenix y Thin2.

\section*{Objetivos}
A continuación se detallan el objetivo general y los objetivos específicos que
se buscan lograr en el desarrollo del proyecto.
% Objetivo general
\subsection*{Objetivo General}
El objetivo general de la pasantía es identificar posibles mejoras para las arquitecturas
de desarrollo ePhoenix y Thin2 utilizadas en el banco BBVA, y una vez
identificadas diseñar e implementar dichas mejoras para enriquecer la experiencia
de los desarrolladores.
% Objetivos específicos
\subsection*{Objetivos Específicos}
\begin{itemize}
  \item Levantar información de uso de la arquitectura ePhoenix.
  \item Una vez identificada una carencia, diseñar una solución para
        mejorar la arquitectura.
  \item Implementar y probar la solución diseñada en ePhoenix.
  \item Levantar información de uso de la arquitectura Thin2.
  \item Una vez identificada una carencia, diseñar una solución para
        mejorar la arquitectura Thin2.
  \item Implementar y probar la solución diseñada en Thin2.
  \item Documentar y divulgar las mejoras implementadas para su
        posterior uso por los equipos de desarrollo del banco.
\end{itemize}

