\chapter*{Introducción}
\label{intro}
\lhead{\emph{Introducción}}
\addcontentsline{toc}{chapter}{Introducción}
% Organización del trabajo
% Se describe brevemente qué se hace en cada capítulo
El presente informe detalla el proceso llevado a cabo
para detectar, diseñar, implementar y probar mejoras a las arquitecturas
ePhoenix y Thin2 utilizadas en el banco español BBVA.

% Descripción del problema, de lo general hacia lo específico

El banco BBVA es una de las entidades líderes de España, con más de 10
millones de clientes y cerca de 30.000 empleados, prestando servicios
financieros a través de su red de 3200 oficinas (\cite{BBVA}). Dentro de
su área de tecnología cuenta con diferentes plataformas y arquitecturas
utilizadas con el objetivo de reducir el tiempo y homogeneizar el desarrollo.

Durante el proyecto se trabajó con dos de las arquitecturas mencionadas anteriormente.
La primera es la arquitectura ePhoenix que le permite a los equipos de desarrollo
realizar procesamientos por lotes y publicar servicios web. La segunda arquitectura
con la que se trabajó fue la arquitectura Thin2, la cual es un conjunto de librerías
basadas en Angular que le permite a los desarrolladores construir eficientemente
aplicaciones web.

\section*{Planteamiento del problema}

Con la finalidad de mantener las arquitecturas utilizadas en el banco lo
más actualizadas posibles es necesario un proceso constante de revisión y mejoras de
las mismas. Este proceso incluye una fase de levantamiento de requerimientos
para identificar fallas o posibles mejoras de las plataformas utilizadas. Luego,
una etapa de diseño de solución que solvente la deuda técnica detectada, para
finalmente pasar al desarrollo, prueba y despliegue de dicha solución.

Para contribuir con la actualización de ambas arquitecturas se debe realizar
una iteración de las etapas mencionadas sobre cada una de las plataformas para
así producir un evolutivo para ePhoenix y Thin2.

Durante el proceso de exploración se detectó que en la arquitectura ePhoenix sólo
se podían realizar conexiones a bases de datos de tipo Oracle, lo cual es una limitante
debido a que buena parte de la información manejada por el banco se encuentra en
bases de datos de otros manejadores, como PostgreSQL y Microsoft SQL Server.

Asimismo se detectó que en el desarrollo de aplicaciones web existen operaciones
repetitivas y tediosas que enlentecen el proceso de desarrollo. Entre estas
operaciones se incluye la creación de tabla de datos, la cual involucra
un proceso bastante repetitivo.

\section*{Justificación e importancia}
Una de las constantes a la hora de desarrollar software es que los
requerimientos pueden cambiar a lo largo del tiempo. Ésto también
es cierto para los empleados encargados de desarrollar aplicaciones
dentro del banco BBVA. Debido a ésto es importante que las arquitecturas
que se utilicen para desarrollar soluciones tecnológicas, en específico
las arquitecturas ePhoenix y Thin2, se mantengan actualizadas y se mejoren
para poder satisfacer todas las necesidades de los desarrolladores.

La ventaja de mantener actualizadas las herramientas utilizadas es que le permite
a las empresas optimizar los procesos de desarrollo tecnológico que se realicen al
eliminar o mejorar los impedimentos que afectan al desarrollo de aplicaciones.

\section*{Objetivos}
A continuación se detallan el objetivo general y los objetivos específicos que
se buscan lograr en el desarrollo del proyecto.
% Objetivo general
\subsection*{Objetivo General}
El objetivo general de la pasantía es identificar posibles mejoras para las arquitecturas
de desarrollo ePhoenix y Thin2 utilizadas en el banco BBVA, y una vez
identificadas diseñar e implementar dichas mejoras para incorporarla dentro de los
estándares de arquitectura de la empresa y así enriquecer la experiencia de los
desarrolladores.
% Objetivos específicos
\subsection*{Objetivos Específicos}
\begin{itemize}
  \item Levantar información de uso de la arquitectura ePhoenix.
  \item Una vez identificada una carencia, diseñar una solución para
        mejorar la arquitectura.
  \item Implementar y probar la solución diseñada en ePhoenix.
  \item Levantar información de uso de la arquitectura Thin2.
  \item Una vez identificada una carencia, diseñar una solución para
        mejorar la arquitectura Thin2.
  \item Implementar y probar la solución diseñada en Thin2.
  \item Documentar y divulgar las mejoras implementadas para su
        posterior uso por los equipos de desarrollo del banco.
\end{itemize}

\section*{Estructura del documento}

El documento se organiza en capítulos de la siguiente forma: la
Introducción al proyecto de Pasantías; en el Capítulo 1 se menciona
el contexto empresarial y el cargo ocupado en la
consultora Everis durante la pasantía; en el Capítulo 2 se explican
los fundamentos teóricos necesarios para la comprensión del problema y
la solución planteada; en el Capítulo 3 se detallan las tecnologías
implicadas en el proceso de desarrollo del proyecto; en el Capítulo 4 se
explica el funcionamiento de la metodología de gestión de proyectos utilizada
(SCRUM); en el Capítulo 5 se describen las actividades realizadas;
finalmente se le dan al lector las Conclusiones y recomendaciones.
