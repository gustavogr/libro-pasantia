\chapter{Entorno Empresarial}
\label{capitulo1}
\lhead{Capítulo 1. \emph{Entorno Empresarial}}

En este capítulo se explica el entorno laborla en el cual el
estudiante desenvolvió su proyecto de pasantía. Se da una descripción
de la empresa Everis, con su visión y valores, y se muestra
como se encuentra organizada la empresa.


\section{Everis}

Everis es una empresa multinacional que se dedica a la consultoría y
\emph{outsourcing} abarcando todos los sectores del ámbito económico.
Actualmente Everis cuenta con alrededor de 19.000
empleados repartidos entre USA, Europa y Latinoamérica.

Desde el año 2014 Everis se unió al grupo NTT DATA, la sexta empresa de
servicios IT del mundo, con 100.000 profesionales y presencia en Asia-Pacífico,
Oriente Medio, Europa, Latinoamérica y Norteamérica.

Pero por encima de todo en Everis se cree en las personas, en su desarrollo integral y en
el talento que representan. En everis apuestan firmemente por el talento, y
su principal objetivo es conseguir un alto rendimiento profesional al crear
un contexto de libertad responsable (\cite{EVERIS}).

\subsection{Visión}

"Everis es una compañía de ámbito mundial, excepcional en términos éticos y
emocionales, liderada por valores y donde cualquier sueño es alcanzable." (\cite{EVERIS}).

\subsection{Valores}

\begin{itemize}
  \item \textbf{Generosidad exigente}: compartimos para hacer.
  \item \textbf{Libertad responsable}: hacemos lo que queremos.
  \item \textbf{Energía creativa}: nos apasiona lo que hacemos.
  \item \textbf{Coherencia}: hacemos lo que decimos.
  \item \textbf{Transparencia}: contamos lo que hacemos. (\cite{EVERIS})
\end{itemize}






% La Figura \ref{usb} muestra el símbolo de nuestra universidad.
% \begin{figure}[h!]
% \centering
% \includegraphics[width=0.4\textwidth]{usb.png}
% \caption[La popular \textit{cebolla}]{La popular \textit{cebolla}, símbolo de la USB.}
% \label{usb}
% \end{figure}

% La Figura \ref{grafo} lo muestra.

% \shorthandoff{<>."}
% \begin{figure}[h]
% \begin{center}
% \begin{tikzpicture}[shorten >=1pt, thick]%[shorten >=1pt,node distance=2cm,>=stealth',thick]
%   \node [shape=circle,fill=black,inner sep=1.5pt,label=below:$s$] (q0) at (0,0) {};
%   \node [shape=circle,fill=black,inner sep=1.5pt,label=below:$1$] (q1) at (2,0) {};
%   \node [shape=circle,fill=black,inner sep=1.5pt,label=below:$t$] (q2) at (4,0) {};
%   \path[->] (q0) edge (q1) (q1) edge (q2);
% \end{tikzpicture}
% \end{center}
% \caption[@descripcionCorta]{@descripcionLarga}
% \label{grafo}
% \end{figure}

% Tabla \ref{tabla:resultados}.

% \begin{table}[h!]
% \begin{center}
% \begin{tabular}{llllll}
% \multicolumn{4}{@{}c}{Nombre del experimento} \\
% \midrule
%               &    éxitos/intentos & tiempo (ms) & espacio (kB) \\
% \midrule
% instancia1          &        28/30 &    23 &       1.7 \\
% instancia2          &        50/70 &    12 &       32.7 \\
% \midrule
% \end{tabular}
% \end{center}
% \caption[Resultados X/Y]{Resultados de X para Y}
% \label{tabla:resultados}
% \end{table}


% En el Apéndice \ref{apendiceA} se encuentra.
