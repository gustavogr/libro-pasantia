% Pagina del acta final
\begin{titlepage}
\begin{center}

% Upper part
\includegraphics[scale=0.4]{usb.png} \\

\textbf{
  \textsc {UNIVERSIDAD SIMÓN BOLÍVAR} \\
  \textsc{DECANATO DE ESTUDIOS PROFESIONALES\\
  COORDINACIÓN DE INGENIERÍA DE LA COMPUTACIÓN}}\\
  \vspace{5mm}
  \textbf{\large Desarrollo de nuevos componentes de las arquitecturas
          tecnológicas utilizadas por BBVA}\\

\vspace{5mm}
INFORME DE PASANTÍA\\
Realizado por: Gustavo Antonio Gutiérrez Rondón\\
Con la asesoría de: Prof. Federico Flaviani\\
\end{center}

\addtotoc{Resumen}
\abstract{
  \addtocontents{toc}{\vspace{1em}}

  En este documento se detallan las actividades realizadas por el autor durante su
  período de pasantías en el desarrollo de mejoras a las arquitecturas tecnológicas
  utilizadas por el banco español BBVA.

  BBVA en búsqueda de estandarizar y de facilitar el desarrollo de aplicaciones
  define cuales son las plataformas a utilizar para implementar dichas aplicaciones.
  Entre estas plataformas se encuentran las arquitecturas ePhoenix, escrita en Java,
  y la arquitectura Thin2, basada en el framework de Javascript Angular.

  Fué responsabilidad del estudiante identificar posibles mejoras a estas arquitecturas,
  y luego diseñar e implementar dichas mejoras en ambas plataformas. Para la gestión del
  trabajo del estudiante se utilizó la metodología ágil de desarrollo
  SCRUM.

  El proyecto de pasantía dió como resultado final la certificación de conexiones
  a bases de datos manejadas con PostgreSQL y Microsoft SQL Server en la arquitectura
  ePhoenix y el desarrollo de un componente reutilizable para crear tablas en la
  arquitectura Thin2.

  % Las palabras clave son generalmente los nombres de áreas de investigación a
  % los cuales está asociado el trabajo. Generalmente son tres o cuatro.
  \noindent \begin{small} \textbf{Palabras clave}: arquitectura, tecnológica, desarrollo, Java, Angular.
  \end{small}
}
\end{titlepage}
