\documentclass[letterpaper, 12pt, oneside]{tesis}

% Paquetes para idioma
\usepackage[spanish]{babel}
\usepackage[utf8]{inputenc}
\usepackage[fixlanguage]{babelbib}

% Otros paquetes instalados
% Básicos
\usepackage{natbib}
\usepackage{enumerate}

% Para dibujar figuras
\usepackage{tikz}

% Para cambiar el color de las letras
\usepackage{color}

% Para incluir código (básico)
\usepackage{verbatim}
\usepackage{fancyvrb}

% Para incluir hipervínculos
\usepackage{hyperref}
\hypersetup{urlcolor=blue, colorlinks=false}

% Para más símbolos matemáticos
\usepackage{amsmath}
\usepackage{amsthm}
\usepackage{amssymb}

% Para colocar teoremas en cajas
\usepackage{mdframed}

% Para texto Lorem Ipsum
\usepackage{blindtext}

% Para tener mas de un título
\usepackage{titling}

% Paquetes locales
% Puedes agregar paquetes locales (archivos .sty) en un subdirectorio % 'paquetes'.
% Utiliza la sintaxis \usepackage{paquetes/nombrePaquete}

% Todas las imágenes se cargan del subdirectorio 'img' por defecto.
\graphicspath{{img/}}

% Sangrías de 3 espacios (3 veces el espacio de la x)
\parindent 3ex

% Interlineado
\setlength{\baselineskip}{1.5pt}

% Interpárrafo
\setlength{\parskip}{16.5pt}

\topmargin 2cm

\renewcommand{\tablename}{Tabla}
\newcommand\listsymbolname{Lista de abreviaturas}

\begin{titlepage}
    \title{\vspace{-2cm} \includegraphics[width=1.2in]{./usb.png} \\[.2cm]
        \large Universidad Simón Bolívar \\
        Decanato de Estudios Profesionales \\
        Coordinación de Ingeniería de la Computación
        \vfill \LARGE \bfseries Desarrollo de nuevos componentes de las arquitecturas
        tecnológicas utilizadas por BBVA \vfill}
    \author{Por: \\
        Gustavo Antonio Gutiérrez Rondón \\[1.2cm]
        \bfseries{INFORME DE PASANTÍAS} \\
Presentado ante la Ilustre Universidad Simón Bolívar \\
como requisito parcial para optar al título de \\
Ingeniero de Computación}
    \date{\bfseries Sartenejas, Octubre de 2017}
\end{titlepage}

\begin{document}
\frontmatter
% Carátula
\maketitle

% Título
\begin{titlepage}
    \title{\vspace{-2cm} \includegraphics[width=1.2in]{./usb.png} \\[.2cm]
        \large Universidad Simón Bolívar \\
        Decanato de Estudios Profesionales \\
        Coordinación de Ingeniería de la Computación
        \vfill \LARGE \bfseries Desarrollo de nuevos componentes de las arquitecturas
        tecnológicas utilizadas por BBVA \vfill}
    \author{Por: \\
        Gustavo Antonio Gutiérrez Rondón \\[1.2cm]
        Realizado con la asesoría de: \\
        Prof. Federico Flaviani \\[1.2cm]
        \bfseries{INFORME DE PASANTÍAS} \\
Presentado ante la Ilustre Universidad Simón Bolívar \\
como requisito parcial para optar al título de \\
Ingeniero de Computación}
    \date{\bfseries Sartenejas, Octubre de 2017}
\end{titlepage}

\maketitle

\setstretch{1.3}

% Se incluye el acta de evaluación, verificar que se corresponda
% con el formato aceptado actualmente por el Decanato.
% Pagina del acta final
\begin{titlepage}
\begin{center}

% Upper part
\includegraphics[scale=0.5]{usb.png} \\

\textsc {\large UNIVERSIDAD SIMÓN BOLÍVAR} \\
\textsc{DECANATO DE ESTUDIOS PROFESIONALES\\
COORDINACIÓN DE INGENIERÍA DE LA COMPUTACIÓN}

\bigskip
\bigskip
\bigskip
\bigskip
\bigskip
\bigskip

% Title
\textsc{ACTA FINAL PROYECTO DE GRADO}

\bigskip
\bigskip

\textsc{\bfseries Desarrollo de nuevos componentes de las arquitecturas
        tecnológicas utilizadas por BBVA}
\bigskip
\bigskip
\bigskip
\bigskip

\begin{minipage}{\textwidth}
\centering
Presentado por: \\
\textsc{\bfseries Gustavo Antonio Gutiérrez Rondón} \\

\bigskip
\bigskip
\bigskip

Este Proyecto de Grado ha sido aprobado por el siguiente jurado examinador: \\

\bigskip
\bigskip

% Despues de cada line coloca el (los) nombre(s) de
% cada uno de los integrantes del jurado.
\line(1,0){200} \\
Federico Flaviani\\

\bigskip
\bigskip

\line(1,0){200} \\
@jurado1\\

\bigskip
\bigskip

\line(1,0){200} \\
@jurado2\\
\end{minipage}

\bigskip
\bigskip
\vfill

% Date/Fecha
{\large \bfseries Sartenejas, @día de @mes de 2017}

\end{center}
\end{titlepage}


% El resumen debe ser de una sola página
\addtotoc{Resumen}
\abstract{
\addtocontents{toc}{\vspace{1em}}

En este documento se detallan las actividades realizadas por el autor durante su
período de pasantías en el desarrollo de mejoras a las arquitecturas tecnológicas
utilizadas por el banco español BBVA.

BBVA en búsqueda de estandarizar y de facilitar el desarrollo de aplicaciones
define cuales son las plataformas a utilizar para implementar dichas aplicaciones.
Entre estas plataformas se encuentran las arquitecturas ePhoenix, escrita en Java,
y la arquitectura Thin2, basada en el framework de Javascript Angular.

Fué responsabilidad del estudiante identificar posibles mejoras a estas arquitecturas,
y luego diseñar e implementar dichas mejoras en ambas plataformas. Para la gestión del
trabajo del estudiante se utilizó la metodología ágil de desarrollo
SCRUM.

El proyecto de pasantía dió como resultado final la certificación de conexiones
a bases de datos manejadas con PostgreSQL y Microsoft SQL Server en la arquitectura
ePhoenix y el desarrollo de un componente reutilizable para crear tablas en la
arquitectura Thin2.

% Las palabras clave son generalmente los nombres de áreas de investigación a
% los cuales está asociado el trabajo. Generalmente son tres o cuatro.
\noindent \begin{small} \textbf{Palabras clave}: arquitectura, tecnológica, desarrollo, Java, Angular.
\end{small}

% Iniciar nueva página luego del resumen
\clearpage
\setstretch{1.3}

% Begin the Dedication page

\setstretch{1.3}  % Return the line spacing back to 1.3

\pagestyle{empty}  % Page style needs to be empty for this page

\dedicatory{
    \textbf{Dedicatoria} \bigskip

    % A mi familia, quienes son la base sobre la que se levantan todos mis logros.
}

\addtocontents{toc}{\vspace{2em}}


% Agradecimientos
\acknowledgements{
\addtocontents{toc}{\vspace{1em}}
}
\clearpage

\pagestyle{fancy}

% Tabla de contenidos o índice
\lhead{\emph{Índice General}}
\tableofcontents

% Estos índices solamente se usan si el libro contiene figuras, tablas y
% algoritmos. Si alguno de estos no se utiliza, comentar o eliminar las líneas
% pertinentes.
\lhead{\emph{Índice de Figuras}}
\listoffigures

\setstretch{1.5}
\clearpage
\lhead{\emph{Lista de Abreviaturas}}
\listofsymbols{ll}
{

    % Aquí van las siglas
    \textbf{API} & \textbf{A}pplication \textbf{P}rogramming \textbf{I}nterface \\
    \textbf{MVC} & \textbf{M}odelo \textbf{V}ista \textbf{C}ontrolador \\
    \textbf{JDBC} & \textbf{J}ava \textbf{D}ata\textbf{B}ase \textbf{C}onnectivity\\
    \textbf{JNDI} & \textbf{J}ava \textbf{N}aming and \textbf{D}irectory \textbf{I}nterface\\
    \textbf{SQL} & \textbf{S}tructured \textbf{Q}uery \textbf{L}anguage \\
    \textbf{HTML} & \textbf{H}yper\textbf{T}ext \textbf{M}arkup \textbf{L}anguage \\
    \textbf{OSGI} & \textbf{O}pen \textbf{S}ervice \textbf{G}ateway \textbf{I}nitiative\\
    \textbf{JSON} & \textbf{J}ava\textbf{S}cript \textbf{O}bject \textbf{N}otation\\
    &\\
    % \hline
    % &\\

    % % Aquí van los símbolos
    % $\iff$ & doble implicación, si y sólo si\\
    % $\Rightarrow$ & implicación lógica\\
    % $[u:=v]$ & sustitución textual de $u$ por $v$
}

%% ----------------------------------------------------------------
% End of the pre-able, contents and lists of things

\mainmatter
\pagestyle{fancy}

% Se incluye el cuerpo de la tesis en este documento.

\chapter*{Introducción}
\label{intro}
\lhead{\emph{Introducción}}
\addcontentsline{toc}{chapter}{Introducción}
% Organización del trabajo
% Se describe brevemente qué se hace en cada capítulo
El presente informe detalla el proceso llevado a cabo
para detectar, diseñar, implementar y probar mejoras a las arquitecturas
ePhoenix y Thin2 utilizadas en el banco español BBVA.

% Descripción del problema, de lo general hacia lo específico

El banco BBVA es una de las entidades líderes de España, con más de 10
millones de clientes y cerca de 30.000 empleados, prestando servicios
financieros a través de su red de 3200 oficinas (\cite{BBVA}). Dentro de
su área de tecnología cuenta con diferentes plataformas y arquitecturas
utilizadas con el objetivo de reducir el tiempo y homogeneizar el desarrollo.

Durante el proyecto se trabajó con dos de las arquitecturas mencionadas anteriormente.
La primera es la arquitectura ePhoenix que le permite a los equipos de desarrollo
realizar procesamientos por lotes y publicar servicios web. La segunda arquitectura
con la que se trabajó fue la arquitectura Thin2, la cual es un conjunto de librerías
basadas en Angular que le permite a los desarrolladores construir eficientemente
aplicaciones web.

\section*{Planteamiento del problema}

Con la finalidad de mantener las arquitecturas utilizadas en el banco lo
más actualizadas posibles es necesario un proceso constante de revisión y mejoras de
las mismas. Este proceso incluye una fase de levantamiento de requerimientos
para identificar fallas o posibles mejoras de las plataformas utilizadas. Luego,
una etapa de diseño de solución que solvente la deuda técnica detectada, para
finalmente pasar al desarrollo, prueba y despliegue de dicha solución.

Para contribuir con la actualización de ambas arquitecturas se debe realizar
una iteración de las etapas mencionadas sobre cada una de las plataformas para
así producir un evolutivo para ePhoenix y Thin2.

Durante el proceso de exploración se detectó que en la arquitectura ePhoenix sólo
se podían realizar conexiones a bases de datos de tipo Oracle, lo cual es una limitante
debido a que buena parte de la información manejada por el banco se encuentra en
bases de datos de otros manejadores, como PostgreSQL y Microsoft SQL Server.

Asimismo se detectó que en el desarrollo de aplicaciones web existen operaciones
repetitivas y tediosas que enlentecen el proceso de desarrollo. Entre estas
operaciones se incluye la creación de tabla de datos, la cual involucra
un proceso bastante repetitivo.

\section*{Justificación e importancia}
Una de las constantes a la hora de desarrollar software es que los
requerimientos pueden cambiar a lo largo del tiempo. Ésto también
es cierto para los empleados encargados de desarrollar aplicaciones
dentro del banco BBVA. Debido a ésto es importante que las arquitecturas
que se utilicen para desarrollar soluciones tecnológicas, en específico
las arquitecturas ePhoenix y Thin2, se mantengan actualizadas y se mejoren
para poder satisfacer todas las necesidades de los desarrolladores.

La ventaja de mantener actualizadas las herramientas utilizadas es que le permite
a las empresas optimizar los procesos de desarrollo tecnológico que se realicen al
eliminar o mejorar los impedimentos que afectan al desarrollo de aplicaciones.

\section*{Objetivos}
A continuación se detallan el objetivo general y los objetivos específicos que
se buscan lograr en el desarrollo del proyecto.
% Objetivo general
\subsection*{Objetivo General}
El objetivo general de la pasantía es identificar posibles mejoras para las arquitecturas
de desarrollo ePhoenix y Thin2 utilizadas en el banco BBVA, y una vez
identificadas diseñar e implementar dichas mejoras para incorporarla dentro de los
estándares de arquitectura de la empresa y así enriquecer la experiencia de los
desarrolladores.
% Objetivos específicos
\subsection*{Objetivos Específicos}
\begin{itemize}
  \item Levantar información de uso de la arquitectura ePhoenix.
  \item Una vez identificada una carencia, diseñar una solución para
        mejorar la arquitectura.
  \item Implementar y probar la solución diseñada en ePhoenix.
  \item Levantar información de uso de la arquitectura Thin2.
  \item Una vez identificada una carencia, diseñar una solución para
        mejorar la arquitectura Thin2.
  \item Implementar y probar la solución diseñada en Thin2.
  \item Documentar y divulgar las mejoras implementadas para su
        posterior uso por los equipos de desarrollo del banco.
\end{itemize}

\section*{Estructura del documento}

El documento se organiza en capítulos de la siguiente forma: la
Introducción al proyecto de Pasantías; en el Capítulo 1 se menciona
el contexto empresarial y el cargo ocupado en la
consultora Everis durante la pasantía; en el Capítulo 2 se explican
los fundamentos teóricos necesarios para la comprensión del problema y
la solución planteada; en el Capítulo 3 se detallan las tecnologías
implicadas en el proceso de desarrollo del proyecto; en el Capítulo 4 se
explica el funcionamiento de la metodología de gestión de proyectos utilizada
(SCRUM); en el Capítulo 5 se describen las actividades realizadas;
finalmente se le dan al lector las Conclusiones y recomendaciones.


% El número de capítulos varía. En mi libro fueron cuatro (sin contar
% introducción y conclusión).
\chapter{Entorno Empresarial}
\label{capitulo1}
\lhead{Capítulo 1. \emph{Entorno Empresarial}}

En este capítulo se explica el entorno laboral en el cual el
estudiante desenvolvió su proyecto de pasantía. Se da una descripción
de la empresa Everis, con su visión y valores, y se muestra
como se encuentra organizada la empresa.


\section{Everis}

Everis es una empresa multinacional que se dedica a la consultoría y
\emph{outsourcing} abarcando todos los sectores del ámbito económico.
Actualmente Everis cuenta con alrededor de 19.000
empleados repartidos entre USA, Europa y Latinoamérica.

Desde el año 2014 Everis se unió al grupo NTT DATA, la sexta empresa de
servicios IT del mundo, con 100.000 profesionales y presencia en Asia-Pacífico,
Oriente Medio, Europa, Latinoamérica y Norteamérica.

Pero por encima de todo en Everis se cree en las personas, en su desarrollo integral y en
el talento que representan. En everis apuestan firmemente por el talento, y
su principal objetivo es conseguir un alto rendimiento profesional al crear
un contexto de libertad responsable (\cite{EVERIS}).

\subsection{Visión}

''Everis es una compañía de ámbito mundial, excepcional en términos éticos y
emocionales, liderada por valores y donde cualquier sueño es alcanzable.'' (\cite{EVERIS}).

\subsection{Valores}

\begin{itemize}
  \item \textbf{Generosidad exigente}: compartimos para hacer.
  \item \textbf{Libertad responsable}: hacemos lo que queremos.
  \item \textbf{Energía creativa}: nos apasiona lo que hacemos.
  \item \textbf{Coherencia}: hacemos lo que decimos.
  \item \textbf{Transparencia}: contamos lo que hacemos. (\cite{EVERIS})
\end{itemize}

\subsection{Organización de la empresa}

Everis España está dividida en líneas operativas y por sectores
dependiendo del tipo de servicio/proyecto o conocimientos que se requieren (\cite{MANUAL}).

Las unidades operativas de Everis España se muestran en la figura \ref{unidades}.
El pasante formaba parte de la unidad operativa de Tecnología.

\begin{figure}[h!]
\centering
\includegraphics[width=\textwidth]{UnidadesOperativas}
\caption[Unidades Operativas de Everis]{Unidades operativas en las que
        está dividida Everis España (\cite{MANUAL}).}
\label{unidades}
\end{figure}


Luego, dentro de cada unidad operativa los consultores se diferencian por categorías.
Las categorías presentes en la unidad operativa de Tecnología se muestran en la
figura \ref{categorias}.
El estudiante durante su pasantía desenvolvió el rol de un consultor de categoría
\emph{SA (Solutions Assistant)}.

\begin{figure}[h!]
\centering
\includegraphics[width=\textwidth]{Cargos-Consultor3}
\caption[Categorías en Tecnología]{Categorías de consultor en la línea operativa
  de Tecnología (\cite{MANUAL}).}
\label{categorias}
\end{figure}

\subsection{Proyecto}
Ejerciendo el cargo de consultor \emph{SA} el estudiante fue asignado al equipo
de arquitectura empresarial \emph{Elara}, el cual está encargado de desarrollar
y mantener las arquitecturas tecnológicas del banco español BBVA.

\chapter{Marco Teórico}
\label{capitulo2}
\lhead{Capítulo 2. \emph{Marco Teórico}}


\chapter{Marco Tecnológico}
\label{capitulo3}
\lhead{Capítulo 3. \emph{Marco Tecnológico}}

En este capítulo se mencionan las herramientas utilizadas para el desarrollo
del proyecto de pasantía. Dichas herramientas son descritas brevemente y se menciona
su uso dentro de la solución planteada.

\section{Java}

Java es un lenguaje de programación de propósito general, concurrente, basado en clases
y orientado a objetos. Está diseñado para ser lo suficientemente sencillo para
que muchos programadores puedan alcanzar fluidez en el lenguaje. (\cite{JAVA})

\section{OSGI}

OSGI es un conjunto de especificaciones que definen un sistema dinámico formado por componentes
para Java. Estas especificaciones permiten un modelo de desarrollo donde las aplicaciones
son dinámicamente compuestas por muchos componentes reutilizables (\cite{OSGI}).

En la figura \ref{layers} se describe el modelo por capas definido por OSGI:

\begin{figure}[h!]
\centering
\includegraphics[width=0.4\textwidth]{layering-osgi}
\caption[Capas de OSGI]{Modelo por capas definido en OSGI. (\cite{OSGI})}
\label{layers}
\end{figure}

La siguiente lista contiene una breve descripción de los términos:
\begin{itemize}
\item \textbf{Bundles}: los Bundles son los componentes de OSGi hechos por los desarrolladores.
\item \textbf{Servicios}: la capa de servicios conecta bundles de una manera dinámica intercambiando objectos Java.
\item \textbf{Ciclo de vida}: el API para instalar, arrancar, detener, actualizar y desinstalar un bundle.
\item \textbf{Módulos}: la capa que define como un bundle puede importar y exportar código.
\item \textbf{Seguridad}: la capa que determina los aspectos de seguridad.
\item \textbf{Ambiente de Ejecución}: define que métodos y clases están disponibles en una plataforma específica.
\end{itemize}


\section{Apache Felix}

Apache Felix es un esfuerzo de la comunidad Apache para implementar especificación de
la plataforma OSGI bajo la licencia Apache. La especificación OSGI es ideal para
cualquier proyecto interesado en los principios de modularidad, desarrollo orientado
a componentes y desarrollo orientado a servicios (\cite{FELIX}).

\section{JDBC}

El \emph{API} JDBC (\emph{Java Database Connectivity}) es el estándar en la industria
para conectividad independiente del manejador de base de datos entre programas escritos en
Java y una gran variedad de bases de datos (\cite{JDBC}).

\section{Oracle}

(\cite{ORACLE}).
La base de datos Oracle es un sistema de manejo de base de datos Objeto-Relacional (\cite{ORACLE})
producido y mantenido por Oracle Corporation.

\section{PostgreSQL}

PostgreSQL es un poderoso sistema de bases de datos Objeto-Relacional. Tiene más de
15 años de desarrollo activo y una arquitectura que se ha ganado la reputación de tener
integridad de datos, confiabilidad y correctitud de datos (\cite{POSTGRE}).

\section{Microsoft SQL Server}
Microsoft SQL Server es un sistema de manejo de bases de datos relacional desarrollado por
Microsoft.

\section{Arquitectura ePhoenix}

Construido sobre la base de Apache Felix, la arquitectura ePhoenix le brinda a los
equipos de desarrollo del BBVA todas las herramientas necesarias para desarrollar programas
que realicen procesamiento por lotes o publiquen servicios web.

Abarcar todas las funcionalidades que ofrece la arquitectura escapa del alcance del
proyecto de pasantías, por lo que se trabajará con los componentes de ePhoenix
que se encargan de realizar las conexiones a bases de datos usando JDBC.

\section{HTML}

HTML, que significa Lenguaje de Marcado para Hipertextos (HyperText Markup Language)
es el elemento de construcción más básico de una página web y se usa para crear y
representar visualmente una página web. Determina el contenido de la página web,
pero no su funcionalidad (\cite{HTML}).

\section{Javascript}

JavaScript es un lenguaje ligero e interpretado, orientado a objetos, más conocido
como el lenguaje de script para páginas web. Es un lenguaje script multi-paradigma, dinámico,
soporta estilos de programación funcional, orientada a objetos e imperativa (\cite{JS}).

\section{Node.js}
Node.js es un entorno de ejecución para JavaScript construido con el motor de JavaScript V8
de Chrome. Node.js usa un modelo de operaciones E/S sin bloqueo y orientado a eventos.
El ecosistema de paquetes de Node.js, npm, es el ecosistema mas grande de librerías
de código abierto en el mundo (\cite{NODE}).

\section{Angular}
Angular es una plataforma escrita en JavaScript que facilita el proceso
de crear aplicaciones para la web. Angular combina plantillas declarativas, injeccion de dependencias
y buenas prácticas integradas para solucionar problemas a la hora de desarrollar (\cite{ANGULAR}).

\section{Arquitectura Thin2}


\chapter{Marco Metodológico}
\label{capitulo4}
\lhead{Capítulo 4. \emph{Marco Metodológico}}

A continuación se procederá a describir el proceso de gestión
del proyecto de pasantía.

El equipo de Everis encargado de mantener las arquitecturas tecnológicas
de BBVA trabaja bajo la metodología de desarrollo ágil SCRUM para el
desarrollo de sus tareas. El estudiante desarrolló el proyecto bajo
esta metodología que define el desarrollo de software como un proceso
incremental e iterativo, con entregables parciales y frecuentes al
cliente (BBVA). La definición de iteraciones, llamadas Sprints, de baja duración
le brindan a esta metodología la ventaja de ser un proceso bastante flexible y
adaptativo a los posibles cambios de requerimientos que puedan ocurrir en el proyecto.

SCRUM se define mediante los elementos descritos en este capítulo.

\section{Sprint}
En SCRUM las iteraciones en las cuales se lleva acabo el desarrollo reciben el nombre
de \emph{Sprint}. Para que la metodología sea eficaz se deben definir los sprints con
una duración relativamente corta, entre 2 o 3 semanas. Ésta es la base sobre la cual
se obtienen las ventajas de flexibilidad y adaptabilidad de la metodología.

En el proyecto de pasantía los Sprint tuvieron una duración de 2 semanas.

\section{Roles}

\subsection{Product Owner}
Es el encargado de representar los intereses del cliente y es el responsable de
que el equipo desarrolle artefactos con valor para el proyecto. Es el encargado
de crear y priorizar las tareas a realizar por el equipo en cada iteración.

En estas pasantías el rol de \emph{Product Owner} lo llevó a cabo el Ing. Rubén
Maldonado.

\subsection{Scrum Master}
El Scrum Master o facilitador es el encargado de velar por que el equipo de
desarrollo no tenga ningún tipo de impedimento a la hora de realizar sus tareas.
Además es el encargado de que se sigan las pautas establecidas por la metodología
de desarrollo.

Para el proyecto el Ing. Felipe Díaz asumió el papel de \emph{Scrum Master}.

\subsection{Equipo}

El equipo de desarrollo es el conjunto de personas encargadas de realizar las tareas
especificadas en cada sprint. La metodología SCRUM no define restricciones sobre como
debe organizarse internamente el equipo, o en otras palabras, se le da la libertad al
equipo de decidir como debe organizarse para cada sprint.

El proyecto de pasantías fue realizado exclusivamente por el estudiante, por lo que
en este caso sería el único integrante del equipo.


\section{Actividades}

Para garantizar el correcto funcionamiento de la metodología tienen lugar frecuentemente
las siguientes actividades:

\subsection{Scrum diario}
Es una reunión corta, con duración máxima de 15 minutos, que se realiza todos los días
del sprint. El objetivo de esta reunión es que cada miembro del equipo exponga cual
es el estado de sus tareas del sprint.

\subsection{Planificación de Sprint}
Al inicio de cada Sprint se deben reunir el Product Owner, el Scrum Master y el equipo
para tomar las decisiones pertinentes sobre el sprint que va a comenzar. En esta actividad
se define:

\begin{itemize}
  \item Cuales son las tareas a realizar.
  \item Definir cuanto tiempo tomará cada tarea.
\end{itemize}

\subsection{Cierre de Sprint}
Una vez finalizado el sprint se revisa cuales de las tareas asignadas en la planificación
fueron terminadas y se presentan los resultados de las tareas terminadas a los posibles
interesados.

\subsection{Retrospectiva de Sprint}

Esta actividad se lleva acabo al finalizar el sprint. Tiene como finalidad revisar como
se desarrolló la metodología en el sprint y se busca mejorar y afinar el proceso.

\chapter{Desarrollo del Proyecto}
\label{capitulo4}
\lhead{Capítulo 4. \emph{Desarrollo del Proyecto}}



\chapter{Conclusiones y Recomendaciones}
\label{conclusiones}
\lhead{\emph{Conclusiones y Recomendaciones}}

El proyecto de pasantías fué culminado exitosamente y dentro del tiempo
estimado, produciendo sendas mejoras para las arquitecturas tecnológicas
Thin2 y ePhoenix usadas en el banco español BBVA. En el caso de ePhoenix
se desarrolló un \emph{bundle} que permitiera a los equipos de desarrollo
realizar conexiones a bases de datos manejadas con PostgreSQL y Mirosoft SQL
Server. Para la arquitectura de aplicaciones web Thin2 se desarrolló un
componente de Angular que le permite a los desarrolladores construir tablas
de datos de una manera sencilla y clara.

La pasantía fue gestionada con el método de desarrollo ágil SCRUM, el
cual le brinda al proyecto una estructura clara y organizada de como realizar
el trabajo, permitiendo a su vez que esta planificación sea flexible y fácilmente
adatable a los cambios de requerimientos que se presentaron durante el proyecto.
El trabajo fue dividido en 10 sprints con una duración de dos semanas cada uno.
Cabe destacar a pesar de que la metodología está diseñada para ser llevada acabo
por equipos de desarrollo integrado por varias personas, en este caso el equipo
solo estuvo integrado por el estudiante debido a que la pasantía fue realizada
de manera individual.

Tanto en ePhoenix como en Thin2 se trabaja bajo la filosofía de desarrollo orientado
a componentes. Al seguir los lineamientos de esta estrategia los componentes desarrollados
presentan una alta cohesión por mantener toda la funcionalidad
relacionada en un sólo componente, y también presentan un bajo acoplamiento debido
a que las piezas de software desarrolladas no dependen del funcionamiento interno de otros
componentes. Éstas son características altamente deseadas en el desarrollo de software, ya que
contribuyen con la mantenibilidad y el fácil entendimiento del código.

Para futuros trabajos se recomienda mantener siempre la documentación de las soluciones
desarrolladas actualizadas. De esta manera las personas que vayan a utilizar o modificar
el software pueden entrar en contexto rápidamente, facilitando la labor y la productividad
de los proyectos. Además se recomienda la utilización de metodologías de desarrollo ágil,
como SCRUM, ya que plantean una manera clara de organizar el trabajo sin sacrificar
la flexibilidad necesaria de un equipo de desarrollo para responder ante posibles cambios de
requerimientos.


% El estilo de la bibliografía es AAAI, definido en el archivo aaai.bst.
\label{Bibliography}
\bibliography{bibliografia}
\lhead{\emph{Bibliografía}}
\bibliographystyle{aaai}
\addtocontents{toc}{\vspace{2em}}

% Apéndices
\appendix
\chapter{Cambios en la interfaz administrativa de ePhoenix}
\label{apendiceA}
\lhead{Apéndice A. \emph{Cambios en la interfaz administrativa de ePhoenix}}

% En los apéndices se incluye cualquier información que no sea esencial para la
% comprensión básica del trabajo, pero provea ejemplos y casos de estudio
% extendidos que permitan un análisis más exhaustivo.

\begin{figure}[h!]
\centering
\includegraphics[width=\textwidth]{InstAdmin-Antes}
\caption[Interfaz de administración original]{Interfaz de administración antes de las modificaciones.}
\label{instadmin:original}
\end{figure}

\begin{figure}[h!]
\centering
\includegraphics[width=\textwidth]{InstAdmin-Despues}
\caption[Interfaz de administración modificada]{Interfaz de administración con desplegable.}
\label{instadmin:modificado}
\end{figure}

\begin{figure}[h!]
\centering
\includegraphics[width=\textwidth]{InstAdmin-Desplegable}
\caption[Interfaz de administración con desplegable]{Interfaz de administración con desplegable.}
\label{instadmin:desplegable}
\end{figure}

\chapter{Pantallazos de la aplicación de ejemplo para Thin2}
\label{apendiceB}
\lhead{Apéndice B. \emph{Pantallazos de la aplicación de ejemplo para Thin2}}

\begin{figure}[h!]
\centering
\includegraphics[width=\textwidth]{author-config}
\caption[Configuración de tabla sencilla]{Configuración de una tabla sin función.}
\label{author:1}
\end{figure}

\begin{figure}[h!]
\centering
\includegraphics[width=\textwidth]{author-html}
\caption[Código html de tabla sencilla]{Código html utilizado para crear una tabla sin función.}
\label{author:2}
\end{figure}

\begin{figure}[h!]
\centering
\includegraphics[width=\textwidth]{author-resultados}
\caption[Tabla sencilla en aplicación]{Resultado de la tabla realizada en la aplicación.}
\label{author:3}
\end{figure}

\begin{figure}[h!]
\centering
\includegraphics[width=\textwidth]{books-config}
\caption[Configuración de tabla con evento]{Configuración de una tabla con una función asociada.}
\label{books:1}
\end{figure}

\begin{figure}[h!]
\centering
\includegraphics[width=\textwidth]{books-html}
\caption[Código html de tabla con evento]{Código html utilizado para crear una tabla con una función asociada.}
\label{books:2}
\end{figure}

\begin{figure}[h!]
\centering
\includegraphics[width=\textwidth]{books-result}
\caption[Tabla con evento en aplicación]{Tabla con función asociada resultante en la aplicación.}
\label{books:3}
\end{figure}

\begin{figure}[h!]
\centering
\includegraphics[width=\textwidth]{book-detail}
\caption[Página al hacer click en la tabla]{Cambio en la aplicación al ejecutar la función asociada a la celda.}
\label{books:4}
\end{figure}

\addtocontents{toc}{\vspace{2em}}

\backmatter

\end{document}
